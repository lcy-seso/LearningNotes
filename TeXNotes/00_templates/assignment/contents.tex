从global向shared memory传输数据和从shared memory向register file传输数据总觉得有某种微妙的不同。

我们首先规定\colorbox{hl}{text}“拷贝”:是将数据$D$ 从place $A$完全地移动到place $B$存储为$D'$。$D'$可以是$D$的一个permutation,也就是说$D'$中的元素和$D$中的元素一一对应,元素数目不变,而顺序可以不相同。

global memory是外存,shared memory,cache和RF是片上存储,后者的capacity总是小于前者。

给定一个AccessMap将其完全地翻译成实现,不是一个单纯的copy macro kernel问题。这里面涉及了(1)分数据(根据capacity决定一次空间执行的数据块的大小);(2)拷贝;(3)时间上重复执行的执行顺序问题;

非常朴素地想,计算过程涉及到数据和多线程两种要素。对于数据,我们总是提供逻辑和物理两种视角:
\begin{enumerate}
    \setlength{\itemsep}{-0.1cm}
    \item high-dimensional array-like的\colorbox{hl}{逻辑视角};能够用\textcolor{blue}{高维逻辑indices寻址},逻辑视角能够改善可编程性,并且隔离与hardware强相关的实现选择问题
    \item \colorbox{hl}{物理视角}
    \item Layout是logical high-dimensional indices和物理寻址之间的映射函数
\end{enumerate}

\begin{figure}[h]
    \centering
    \includegraphics[width=0.8\textwidth]{figures/shared_2_rf_with_ldmatrix.pdf}
    \caption{使用ldmatrix指令从shared memory向regisger file加载数据。}
\end{figure}

使用ldmatrix从shared memory加载数据到每个线程thread local的寄存器,warp中的每32线程构成一个 $2 \times 2$的线程tile,每个tile内部8线程,
\textcolor{red}{调用ldmatrix的时候每个线程都需要传入一个shared memory指针},然后单线程读取shared memory中连续128 bits。ldmatrix一次执行最大读取 $16 \times 16$大小的半精度矩阵。

在实现中,每个线程都需要正确地计算出自己要读取的shared memory位置的指针偏移。

ldmatrix的一次执行32个线程能一次性读取$16 \times 16$大小的$2D$ tile,一次执行单线程数据tile大小$1 \times 8$,如果将$n$次执行ldmatrix的结果都保留在thread local的寄存器上,单线程数据块的大小是$1 \times \left( n \times 8 \right)$。

所以我们将目标的layout配置成$(1, n*8)$

\newpage

\begin{enumerate}
    \setlength{\itemsep}{-0.1cm}
    \item 第一个嵌套层级:一个shared memory数据块要转化为多次$(m, n)$次对copy\_2d\_tile\_s2r在时间上的调用,要用for循环issue出去。每个处理一个小分块 $\mathcal{T}$。
    \item 第二个嵌套层级:一个copy\_2d\_tile\_s2r
\end{enumerate}