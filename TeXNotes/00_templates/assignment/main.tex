\documentclass{article}

\usepackage {amsmath, amsfonts, stmaryrd, amssymb} % Math packages
\usepackage {cite}
\usepackage[rgb, dvipsnames] {xcolor}
\usepackage{soul}
\usepackage{color}

\usepackage {enumerate} % Custom item numbers for enumerations
\usepackage{subfigure}
\usepackage[ruled] {algorithm2e} % Algorithms

\usepackage[framemethod = tikz] {mdframed} % Allows defining custom boxed/framed environments
\usepackage{tcolorbox}

\usepackage[colorlinks, citecolor=red]{hyperref}
\hypersetup{
    colorlinks=true,
    linkcolor=blue,
    filecolor=blue,      
    urlcolor=blue,
    citecolor=blue,
}
\usepackage{url}

\usepackage{booktabs}

\usepackage{mathtools, nccmath}
\usepackage{cleveref}
% \usepackage{showframe}
% \renewcommand{\ShowFrameLinethickness}{0.3pt}
%-------------------------------------------------------------------------------
%  Code snippets.
%-------------------------------------------------------------------------------
\usepackage {listings} % File listings, with syntax highlighting
\definecolor{forestgreen}{RGB}{34,139,34}
\definecolor{blue1}{RGB}{24,116,205}
\definecolor{aliceblue}{RGB}{240,248,255}
\definecolor{maroon}{RGB}{176,48,96}
\definecolor {dkgreen}{RGB}{0,139,139}
\definecolor{byzantium}{rgb}{0.44,0.16,0.39}
\definecolor{whitesmoke}{RGB}{245,245,245}
\definecolor{grey}{RGB}{128,128,128}
\definecolor{tomato}{RGB}{139,26,26}
\definecolor{commentsColor}{rgb}{0,128,128}
\definecolor{keywordsColor}{rgb}{0.0,0.0,0.64}
\definecolor{stringColor}{rgb}{0.558215,0.0,0.135316}
\definecolor{snow}{RGB}{207,207,207}
\definecolor{NavajoWhite}{RGB}{255,222,173}
\definecolor{LightSkyBlue1}{RGB}{176,226,255}
\definecolor{DarkSeaGreen}{RGB}{180,238,180}
\definecolor{LightSkyBlue3}{RGB}{159,182,205}

\lstset{ %
  backgroundcolor=\color{aliceblue},
  basicstyle=\scriptsize,
  breakatwhitespace=false,
  breaklines=true,
  captionpos=b,
  commentstyle=\color{commentsColor}\textit,
  deletekeywords={...},
  escapeinside={\%*}{*)}, 
  extendedchars=true,
  frame=lrtb,
  keepspaces=true,
  keywordstyle=\color{keywordsColor}\bfseries,
  backgroundcolor=\color{aliceblue},
  language=Python,
  otherkeywords={*,...},
  numbers=left, 
  numbersep=5pt,
  xleftmargin=1em,
  numberstyle=\tiny\color{grey},
  showspaces=false,
  showstringspaces=false,
  showtabs=false,
  stepnumber=1,
  stringstyle=\color{bluebell},
  tabsize=2,
  title=\lstname,
  columns=fixed
}

\lstdefinelanguage{code_example}{
  keywords={},
  keywordstyle=\color{orchid}\bfseries,
  keywords=[2]{foreach, in, return, zip, function},
  keywordstyle=[2]\color{blue1}\bfseries,
  sensitive=false,
  comment=[l]{//},
  morecomment=[s]{/*}{*/},
  commentstyle=\color{byzantium}\ttfamily,
  stringstyle=\color{dkgreen},
  morestring=[b]',
  morestring=[b]",
  escapeinside={(:}{:)}
}

\lstdefinelanguage{code_example2}{
  keywords={},
  keywordstyle=\color{orchid}\bfseries,
  keywords=[2]{for, do, endfor},
  keywordstyle=[2]\color{blue1}\bfseries,
  sensitive=false,
  comment=[l]{//},
  morecomment=[s]{/*}{*/},
  commentstyle=\color{byzantium}\ttfamily,
  stringstyle=\color{dkgreen},
  morestring=[b]',
  morestring=[b]",
  escapeinside={(:}{:)}
}


%-------------------------------------------------------------------------------
%  DOCUMENT MARGINS
%-------------------------------------------------------------------------------
\usepackage{geometry} % Required for adjusting page dimensions and margins

\geometry{
  paper=a4paper, % Paper size, change to letterpaper for US letter size
  top=2.5cm, % Top margin
  bottom=3cm, % Bottom margin
  left=2.5cm, % Left margin
  right=2.5cm, % Right margin
  headheight=14pt, % Header height
  footskip=1.5cm, % Space from the bottom margin to the baseline of the footer
  headsep=1.2cm, % Space from the top margin to the baseline of the header
  %showframe, % Uncomment to show how the type block is set on the page
}

%-------------------------------------------------------------------------------
%  FONTS
%------------------------------------------------------------------------------
\usepackage[utf8]{inputenc} % Required for inputting international characters
\usepackage[T1]{fontenc} % Output font encoding for international characters

\usepackage{XCharter} % Use the XCharter fonts

%-------------------------------------------------------------------------------
%  NUMBERED QUESTIONS ENVIRONMENT
%-------------------------------------------------------------------------------

% Usage:
% \begin{question}[optional title]
%  Question contents
% \end{question}

\mdfdefinestyle{question}{
  innertopmargin=1.2\baselineskip,
  innerbottommargin=0.8\baselineskip,
  roundcorner=5pt,
  nobreak,
  singleextra={%
    \draw(P-|O)node[xshift=1em,anchor=west,fill=white,draw,rounded corners=5pt]{%
    Question \theQuestion\questionTitle};
  },
}

\newcounter{Question} % Stores the current question number that gets iterated
                      % with each new question

% Define a custom environment for numbered questions
\newenvironment{question}[1][\unskip]{
  \bigskip
  \stepcounter{Question}
  \newcommand{\questionTitle}{~#1}
  \begin{mdframed}[style=question]
}{
  \end{mdframed}
  \medskip
}

%-------------------------------------------------------------------------------
%  INFORMATION ENVIRONMENT
%-------------------------------------------------------------------------------

% Usage:
% \begin{info}[optional title, defaults to "Info:"]
%   contents
% \end{info}

\mdfdefinestyle{info}{%
  topline=false, bottomline=false,
  leftline=false, rightline=false,
  nobreak,
  singleextra={%
    \fill[black](P-|O)circle[radius=0.4em];
    \node at(P-|O){\color{white}\scriptsize\bf i};
    \draw[very thick](P-|O)++(0,-0.8em)--(O);%--(O-|P);
  }
}

% Define a custom environment for information
\newenvironment{info}[1][Info:]{ % Set the default title to "Info:"
  \medskip
  \begin{mdframed}[style=info]
    \noindent{\textbf{#1}}
}{
  \end{mdframed}
} % Include the file specifying the document structure and custom commands
\title{My Document}
\author{Ying Cao}
\date{\today}

\begin{document}
\bibliographystyle{plain}

\maketitle % Print the title
\tableofcontents

\noindent
\linespread{1.2}
\selectfont
\setlength{\topskip}{0ex}
\setlength{\parskip}{1ex}
\setlength{\lineskip}{1em}

%---------------------------------------------------------------
% unnumbered section
%---------------------------------------------------------------

\section{Introduction}

This is the introduction. 

Paper \cite{lamport1974parallel} studies the compilation techniques for a kind of loop program transformation
targeting multiprocessor computers.

\begin{figure}[ht]
  \centering
  \includegraphics[scale=0.4]{figures/figure1.png}
  \caption{this is a figure demo}
  \label{fig:label}
\end{figure}

% Math equation/formula
\begin{equation}
I = \int_{a}^{b} f(x) \; \text{d}x.
\end{equation}

\begin{info} % Information block
This is an interesting piece of information, to which the reader should pay special attention.
\end{info}

\section{Section title} % Numbered section

This is the numbered section.

\subsection{Subsection title}

This is the numbered sub section.

%------------------------------------------------------------------
% Numbered question, with subquestions in an enumerate environment
%------------------------------------------------------------------
\begin{question}
    This is a question.

  % Subquestions numbered with letters
  \begin{enumerate}[(a)]
    \item Do this.
    \item Do that.
    \item Do something else.
  \end{enumerate}
\end{question}

%------------------------------------------------------------------
% Algorithm
%------------------------------------------------------------------
\subsection{Algorithmic issues}


\begin{center}
  \begin{minipage}{0.5\linewidth} % Adjust the minipage width to accomodate for the length of algorithm lines
    \begin{algorithm}[H]
      \KwIn{$(a, b)$, two floating-point numbers}  % Algorithm inputs
      \KwResult{$(c, d)$, such that $a+b = c + d$} % Algorithm outputs/results
      \medskip
      \If{$\vert b\vert > \vert a\vert$}{
          exchange $a$ and $b$ \;
      }
      $c \leftarrow a + b$ \;
      $z \leftarrow c - a$ \;
      $d \leftarrow b - z$ \;
      {\bf return} $(c,d)$ \;
      \caption{\texttt{FastTwoSum}} % Algorithm name
      \label{alg:fastTwoSum}   % optional label to refer to
    \end{algorithm}
  \end{minipage}
\end{center}

Descriptions of the algorithm.

%------------------------------------------------------------------
% Numbered question, with an optional title
%------------------------------------------------------------------
\begin{question}[\itshape (with optional title)]
    Describe your question.
\end{question}

Desciptions.

\section{Implementation}

Describe your implementation here.

% code snippets.
\begin{lstlisting}[language=Python]
#! /usr/bin/python

import sys
sys.stdout.write("Hello World!\n")
\end{lstlisting}

Explanations of the implementation.

\bibliography{references.bib}
\end{document}
