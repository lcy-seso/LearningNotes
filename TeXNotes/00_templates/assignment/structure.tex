\usepackage {amsmath, amsfonts, stmaryrd, amssymb} % Math packages
\usepackage {cite}
\usepackage[rgb, dvipsnames] {xcolor}

\usepackage {enumerate} % Custom item numbers for enumerations

\usepackage[ruled] {algorithm2e} % Algorithms

\usepackage[framemethod = tikz] {mdframed} % Allows defining custom boxed/framed environments

%-------------------------------------------------------------------------------
%  Code snippets.
%-------------------------------------------------------------------------------
\usepackage {listings} % File listings, with syntax highlighting
\definecolor {dkgreen} {RGB} {0, 139, 139}
\lstset { % 
language = Python,  % the language of the code
    basicstyle = \footnotesize,  % the size of the fonts that are used for the code
    numbers = left,  % where to put the line - numbers
    numberstyle = \tiny\color {gray},  % the style that is used for the line - numbers
    stepnumber = 1,  % the step between two line - numbers. If it's 1, each line will be numbered
    numbersep=5pt,                  % how far the line-numbers are from the code
    backgroundcolor=\color{white},  % choose the background color. You must add \usepackage{color}
    showspaces=false,               % show spaces adding particular underscores
    showstringspaces=false,         % underline spaces within strings
    showtabs=false,                 % show tabs within strings adding particular underscores
    frame=single,                   % adds a frame around the code
    rulecolor=\color{black},        % if not set, the frame-color may be changed
                                    % on line-breaks within not-black text (e.g. commens (green here))
    tabsize=2,                      % sets default tabsize to 2 spaces
    captionpos=b,                   % sets the caption-position to bottom
    breaklines=true,                % sets automatic line breaking
    breakatwhitespace=false,        % sets if automatic breaks should only happen at whitespace
    title=\lstname,                 % show the filename of files included with \lstinputlisting;
                                    % also try caption instead of title
    keywordstyle=\color{blue},      % keyword style
    commentstyle=\color{dkgreen},   % comment style
    stringstyle=\color{gray},       % string literal style
}

%-------------------------------------------------------------------------------
%  DOCUMENT MARGINS
%-------------------------------------------------------------------------------
\usepackage{geometry} % Required for adjusting page dimensions and margins

\geometry{
  paper=a4paper, % Paper size, change to letterpaper for US letter size
  top=2.5cm, % Top margin
  bottom=3cm, % Bottom margin
  left=2.5cm, % Left margin
  right=2.5cm, % Right margin
  headheight=14pt, % Header height
  footskip=1.5cm, % Space from the bottom margin to the baseline of the footer
  headsep=1.2cm, % Space from the top margin to the baseline of the header
  %showframe, % Uncomment to show how the type block is set on the page
}

%-------------------------------------------------------------------------------
%  FONTS
%------------------------------------------------------------------------------
\usepackage[utf8]{inputenc} % Required for inputting international characters
\usepackage[T1]{fontenc} % Output font encoding for international characters

\usepackage{XCharter} % Use the XCharter fonts

%-------------------------------------------------------------------------------
%  NUMBERED QUESTIONS ENVIRONMENT
%-------------------------------------------------------------------------------

% Usage:
% \begin{question}[optional title]
%  Question contents
% \end{question}

\mdfdefinestyle{question}{
  innertopmargin=1.2\baselineskip,
  innerbottommargin=0.8\baselineskip,
  roundcorner=5pt,
  nobreak,
  singleextra={%
    \draw(P-|O)node[xshift=1em,anchor=west,fill=white,draw,rounded corners=5pt]{%
    Question \theQuestion\questionTitle};
  },
}

\newcounter{Question} % Stores the current question number that gets iterated
                      % with each new question

% Define a custom environment for numbered questions
\newenvironment{question}[1][\unskip]{
  \bigskip
  \stepcounter{Question}
  \newcommand{\questionTitle}{~#1}
  \begin{mdframed}[style=question]
}{
  \end{mdframed}
  \medskip
}

%-------------------------------------------------------------------------------
%  INFORMATION ENVIRONMENT
%-------------------------------------------------------------------------------

% Usage:
% \begin{info}[optional title, defaults to "Info:"]
%   contents
% \end{info}

\mdfdefinestyle{info}{%
  topline=false, bottomline=false,
  leftline=false, rightline=false,
  nobreak,
  singleextra={%
    \fill[black](P-|O)circle[radius=0.4em];
    \node at(P-|O){\color{white}\scriptsize\bf i};
    \draw[very thick](P-|O)++(0,-0.8em)--(O);%--(O-|P);
  }
}

% Define a custom environment for information
\newenvironment{info}[1][Info:]{ % Set the default title to "Info:"
  \medskip
  \begin{mdframed}[style=info]
    \noindent{\textbf{#1}}
}{
  \end{mdframed}
}