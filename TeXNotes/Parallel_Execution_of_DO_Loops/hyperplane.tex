\subsection {Definitions}

\begin{itemize}
  \item VAR:a program array variable.
  \item \textbf {\textit {occerence}}:any appearance of VAR in the loop body.
  \item \textbf {\textit {generation}}/\textbf {\textit {use}}:VAR appears on the left - hand side of an assignment statement;
  Such an occurrence is called \textit {generation}; otherwise, a \textit {use}.
  \begin{enumerate}
    \item generations modify values of array elements which uses do not.
    \item an occurrence is a \textit {generation}or \textit {use}.
  \end{enumerate}
  \item \textbf {\textit {occurrence mapping}}:$T_f:\zeta \rightarrow Z^d$ where $f$ is
  an occerence. $d$ is the dimensionality of the array to access.
  \begin{enumerate}
    \item the occurrence mapping maps points in index set $\zeta$ to array indices.
    \item the occurrence mapping relates time and space.
  \end{enumerate}
\end {itemize}

\subsection {Assumptions}

Following assumptions are made to the loop body:
\begin {itemize}
\item [(A1)] It contains no I/O statement.
\item [(A2)] It contains no transfer of control to any statement outside the loop.
\item [(A3)] It contains no subroutine or function call which can modify data.
\item [(A4)] Any occurrence in the loop body of a generated variable VAR is of the
form $\text {VAR}(e^1, ..., e^\tau)$, where each $e^i$ is an expression not containing any
generated variable.
\end {itemize}

\subsection {Formulation of the problem}

The hyperplane method formulates performing the rewriting procedure as constructing a
\href {https://en.wikipedia.org/wiki/Injective_function}{one-to-one} linear mapping
$J:\mathbf {Z}^n \rightarrow \mathbf {Z}^n$ of the form:
\begin {equation}
\begin {aligned}
J[(I^1, ..., I^n)] &= \left(\sum^ {n}_ {j = 1}a^1_jI^j, ..., \sum^ {n}_ {j = n}a^n_jI^j \right)\\ &= (J^1, ..., J^n)\label {eq1}
\end {aligned}
\end {equation}

In fact, for the \textit {finite - dimensional vector spaces with a defined basis},
a vector spaces we're mostly interested in, any linear mapping can be represented
by a matrix $T$ that is multiplied by the input vector\cite{EliBendersky}. So, we can
write equation (\ref{eq1}) in the matrix form.

\begin{equation}
\begin{aligned}
  J[(I^1,...,I^n)] &=
  \begin{pmatrix}
    a_{1,1} & a_{1,2} & \cdots & a_{1,n} \\
    a_{2,1} & a_{2,2} & \cdots & a_{2,n} \\
    \vdots  & \vdots  & \ddots & \vdots  \\
    a_{n,1} & a_{n,2} & \cdots & a_{n,n}
  \end{pmatrix}
  \begin{pmatrix}
    I_{1}\\
    I_{2}\\
    \vdots\\
    I_{n}\\
  \end{pmatrix}
\end{aligned}
\end{equation}

Once the one-to-one mapping $J$ is constructed, we then choose:
\begin{itemize}
  \item [1)]
  the $\lambda^i$, $\mu^i$ and $\mathcal{S}_{J^1,...,J^k}$ to assure that the
  index set $\zeta$ of loop (\ref{loop2}) equals $J(\zeta)$.
  \item [2)]
  write the loop body of loop (\ref{loop2}).
\end{itemize}

To perform rewriting, two important questions should be answered:
\begin{question}
  \begin{itemize}
  \item [1]
  Under what conditions, the rewritten loop (\ref{loop2}) is equivalent to the given
  loop (\ref{loop1})?
  \item [2]
  How to construct the one-to-one linear mapping $J$ (or construct the matrix $T$)?
  \end{itemize}
\end{question}

\subsection{Concurrent executions of the loop body}

Define mapping $\pi :\mathbf{Z}^n\rightarrow\mathbf{Z}^k$ by $\pi[(I^1,...,I^n)]=(J^1,...,J^k)$:

\begin{itemize}
\item mapping $\pi(P)$ contains the first $k$ coordinates of $J(P)$, which are sequential loops.
\item the set defined by $\left\{ P:\pi(P)=\text{constant}\in \mathbf{Z}^k \right\}$
are parallel $(n-k)$-dimensional planes in $\mathbf{Z}^n$.
Loop body is executed concurrently for elements of $\zeta$ lying on these sets.
\item these sets are parallel $(n-k)$-dimensional planes in
$\mathbf{Z}^n$, hence the name "hyperplane method".
\end{itemize}

\subsection{Conditions for a legal rewriting}

\begin{info}[\textbf{\textit{Sufficient condition}} for loop (\ref{loop2})
to be equivalent to loop (\ref{loop1})]
\begin{itemize}
\item [(C1)]
  For {\color{red}{\textit{\textbf{every}}}} variable, and {\color{red}{\textit{\textbf{every}}}}
  ordered pair of occurrences $f$,$g$ of that variable,
  at least one of which is a generation: if $T_f(P)=T_g(Q)$ for $P, Q \in \zeta$
  with $P<Q$, then $\pi$ must satisfy the relateion $\pi(P)<\pi(Q)$.
\end{itemize}
\end{info}
For most loops, (C1) is also a necessary condition.

However, (C1) requires to consider many pairs of points $P$, $Q$ in $\zeta$.
To address this problem and ease the analysis, the author uses a set descriptor
$\langle f,g \rangle$ rather than directly considering all the pairs of $P$,$Q$.

Define $\langle f,g \rangle$ a subset of $\mathbf{Z}^n$ by:
$$\langle f,g \rangle=\left\{X:T_f(P)=T_g(P+X)\quad \text{for some } P \in \mathbf{Z}^n\right\}$$

Set $\langle f, g \rangle$ defines pairs of \textit{use} and \textit{generation} that
accesses the same memory location. Though this notation is not implicitly called
dependence vectors in this paper, I think it is $\mathbf{X}$ essentially the
dependence vector in later research works. Then, we have a more strigent rule:

\begin{info}[Rule for loop (\ref{loop2}) to be equivalent to loop (\ref{loop1})]
\begin{itemize}
\item [(C2)]
  For every variable, and every ordered pair of occurrences $f$,$g$ of that variable,
  at least one of which is a generation: for every $\mathbf{X} \in \langle f, g \rangle$
  with $\mathbf{X}>\mathbf{0}$, $\mathbf{\pi}$ must satisfy $\mathbf{\pi}(\mathbf{X})>\mathbf{0}$.
\end{itemize}
\end{info}

To guarantee that it is feasible to find a mapping $\pi$ which satisfies (C2),
the author futher make some restriction (see (A5) in the paper) on the forms of
variable occurrences. This restriction actually leads to \textbf{\textit{constant dependence vectors}}.

\subsection{The hyperplane theorem}

\subsubsection{The existence of $\pi$}
C2 gives a set of constraints on the mapping $\pi: \mathbf{Z}^n \rightarrow \mathbf{Z}$,
and the Hyperplane Theorem proves \textit{\textbf{the existence}} of a $\pi$
satisfying those constraints. The proof can also serve as an algorithm for
constructing a mapping $\pi$, but it is not always the optimal.

\newtheorem*{theorem}{HYPERPLANE CONCURRENCY THEOREM}\label{hh}

\begin{theorem}
Assume that loop (\ref {loop1})satisfies (A1) - A(5), and that none of the
index variables {\color{red}$I^2, ..., I^n$} are missing variable. Then it
can be rewritten in the form of (\ref {loop2})for $k = 1$. Moreover, the mapping
$J$ used for the rewriting can be chosen to be independent of the index set $\zeta$.
\end{theorem}

\textbf{Notations}
\begin{itemize}
\item The mapping $\pi$ is defined by:
    $$\pi[(I^1, ...,I^n)] = a_1I^1 + ... + a_nI^n$$
\item $\mathcal{\Theta} = \left\{\mathbf{X}_r\right\},r = 1,...,N$,
$\mathbf{X}_r > \mathbf{0}$ is all the elements of $\langle f, g \rangle$
referred to in (C2). Since $I^1$ is the only index variable that may be mising,
$X_r = (x_r^1,...,x_r^n)$ or $X_r = (+,x_r^2,...,x_r^n)$.
\item $\mathcal{\Theta}_j = \left\{\mathbf{X}_r: x_r^1=...=x_r^{j-1}=0,x_r^j \ne 0 \right\}$.
$\mathcal{\Theta}_j$ is the set of all $\mathbf{X}_r$ whose $j$th
coordinate is the left-most nonzero one. Each $\mathbf{X}_r$ is an element of some
$\mathcal{\Theta}_j$.
\end{itemize}

\begin{proof}
If we can always construct the mapping $\pi:\mathbf{Z}^n \rightarrow \mathbf{Z}$,
and from $\pi$ obtrain the one-one mapping $J$, then the theorem is proved.

\begin{enumerate}
\item Construct the mapping $\pi: \mathbf{Z}^n \rightarrow \mathbf{Z}$ for each
$\mathbf{X}_r \in \Theta$.

  \begin{enumerate}

  \item replace each $X_r = (+,x_r^2,...,x_r^n)$ by $(1,x_r^2,...,x_r^n)$. This is
  because we require $\pi[(+, x_r^2,...,x_r^n)] > 0$, then it is sufficient to consider
  replacing $x$ with the smallest positive integer 1.
  \item for $j=n,n-1,...,1$, let $a_j$ be the smallest nonnegative integer such
  that $a_j x_r^j + ... + a_n x_r^n > 0$ for each $X_r = (0,...,0,x_r^j,...,x_r^n) \in \Theta_j$.

  \end{enumerate}

\item construct $J[(I^1,...,I^n)] = (\pi[(I^1,...,I^n)],...)$. If $a_j = 1$,
define $J$ as follows: for each $l \ge 2$, let $J^k$ equal some distinct $I^{l_k}$
with $l_j \ne j$.

\end{enumerate}
\end{proof}


\subsubsection{The optimality of $\pi$}

The hyperplane theorem proves the existence of $\pi$. So far, there is still
a problem left, the optimal $\pi$.

$k=1$ means the outermost loop is sequential while all the inner loops are parallelable.
Therefore, it is reasonable to minimize the number of steps in the outer \textit{\textbf{DO}}
$J^1$ loop(\ref{loop2}). This means minimizing $\mu^1 - \lambda^1$ in loop(\ref{loop2}).

Setting $M^i = \mu^i - l^i$, it is easy to see that $\mu^1 - \lambda^1$ equals:

\begin{equation} \label{opt}
M^1|a_1|+...+M^n|a_n|
\end{equation}

Thus, finding an optimal $\pi$ is reduced to the integer programming problem:
\textit{\textbf{find integers $a_1,...,a_n$ satisfying inequalities given by (C2)
which minimize expression \ref{opt}}}.

This problem is to find solution to the linear
{\href{https://books.google.co.jp/books?hl=zh-CN&lr=&id=QugvF7xfE-oC&oi=fnd&pg=PP1&dq=mordell+diophantine+equations&ots=KQngn0xTIB&sig=TRaDlam9XGLmO5l-PuFtZE8X0xw#v=onepage&q&f=false}{Diophantine equation}}\cite{Diophantinewiki,mordell1969diophantine}.
For some linear Diophantine equations, it is algorithmatically solvable, but there
are no practical method of finding a solution in the general case.

\begin {info}
Below theorem is from \cite{irigoin1988supernode}. It is referred as an important
conclusion in \cite{wolf1991loop}. I just think
the conclusion below further improves the \textbf{hyperplane concurrency theorem} by:

\begin{enumerate}
  \item propose the concept "fully permutable loop" which gives a more strigent
  conditions of loop programs are able to obtain the optimal parallelism. The
  concept "fully permutable loop" is also the foundation for tiling.
  \item give the transformation that leads to the optimal parallelism for "fully
  permutable loop".
\end{enumerate}

\fbox{\parbox{\textwidth}{\textsl {A nest of $n$ fully permutable loops can be
transformed to code containing at least $n-1$ degress of parallelism. In the
degenerate case when no dependences are carried by these $n$ loops, the degree
of parallelism is $n$.}}}
\end {info}

