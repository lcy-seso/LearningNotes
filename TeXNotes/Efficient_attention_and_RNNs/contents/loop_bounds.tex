\subsection{循环倾斜后循环边界的上下界的求解}

约定loop nest深度的计数从循环最外层向循环体递减,也就是最外层层数最深,循环体深度记为$d = 0$。

图\ref{cond_branchs}中数据流依赖最复杂的控制区域$s_3$用等价的imperative for loop写出来如下:
\begin{lstlisting}[language=cplus]
for (:$i$:) in range (:$[0, N)$:):  // map, 全并行
  for (:$j$:) in range (:$[1, D)$:):  // scan,控制深度2携带数据流依赖
    for (:$k$:) in range (:$1, L)$:):  // scan,控制深度1携带数据流依赖
        (:\textit{UDF}:)(...)
\end{lstlisting}

以上嵌套循环程序的迭代空间是一个三维立方体,可以用如下不等式表示:

\begin{equation}
\begin{aligned}
0 &\le i < N \\
1 &\le j < D \\
1 &\le k < L
\end{aligned}\label{eq:bounds}
\end{equation}

如果我们把迭代空间看做是一个三维空间中的整数点集合,整个循环体看做一个statement,记作$s_x$。x是迭代空间中的一个整数点,也就是$s$的一次执行。
$s_{i_1, j_1, k_1}$写,$s_{i_2, j_2, k_2}$读同一个buffer位置,这两次执行之间存在数据流依赖。
图\ref{fig:dataflow-stacked-rnn}直观地揭示了另一个关于计算过程数据流依赖情况的事实:上面这个loop nest在迭代空间中携带了\textbf{\textcolor{blue}{两类}}数据流,读和写之间的距离是:
$d_1 = [0, 1, 0]$和$d_2 = [0, 0, 1]$。
loop nest中的数据流依赖可以用如下dependence distance vector矩阵$M$描述:

\begin{equation*}
    \begin{bmatrix}
        \vec{d_1} & \vec{d_2}
    \end{bmatrix} = 
    \begin{bmatrix}
        0 & 0 \\ 1 & 0 \\ 0 & 1
    \end{bmatrix}
\end{equation*}

我们要求解的一个调度问题是,给定$M$作为约束,求解一个变化矩阵$T$,能够让最外层循环携带所有数据流依赖。
这里我们暂时忽略矩阵$T$是如何得到的,只需要先记住$T$不唯一,$T$的求解是一个well-studied问题,有完善的理论和工具。
下面直接给出$T$的一种可能结果:

\begin{equation*}
    \delta_A =
    \begin{bmatrix}
        0 & 1 & 1 \\ 1 & 0 & 0 \\ 0 & 1 & 0
    \end{bmatrix}
    \begin{bmatrix}
        i \\ j \\ k
    \end{bmatrix} =
    \begin{bmatrix}
        j+k \\ i \\ j
    \end{bmatrix} =
    \begin{bmatrix}
        m \\ n \\ p
    \end{bmatrix}
\end{equation*}

于是有:
\begin{equation}
    \begin{bmatrix}
        i \\ j \\ k
    \end{bmatrix} =
    \begin{bmatrix}
        n \\ p \\ m-p
    \end{bmatrix}\label{eq:skew}
\end{equation}

我们会得以$m$, $n$, $p$为迭代变量的变化后的循环程序。循环体执行时,在遇到$i$,$j$,$k$的地方,带入\eqref{eq:skew}。
那么下一个问题是,变化后的循环边界如何求解?(\textcolor{brown}{Fourier-Motzkin Elimination算法就是专门解决这个问题的方法}。isl这样的polyhedral工具都集成了这一算法)

\begin{lstlisting}[language=cplus]
for (:$m$:) in range (:$\textcolor{red}{?}$:):  // map, 全并行
  for (:$n$:) in range (:$\textcolor{red}{?}$:):  // scan,控制深度2携带数据流依赖
    for (:$p$:) in range (:$\textcolor{red}{?}$:):  // scan,控制深度1携带数据流依赖
        (:\textit{UDF}:)(...)
\end{lstlisting}

这里我们手算一下变化后的循环边界。

把\eqref{eq:skew}带入\eqref{eq:bounds}我们会得到:

\begin{align}
\textcolor{dkgreen}{0} &\textcolor{dkgreen}{\le n < N} \\
1 &\le p < D  \label{eq:bound-d}\\
1 &\le m - p < L \label{eq:bound-l}
\end{align}

最外层的循环边界必须是常数。将\eqref{eq:bound-d}和\eqref{eq:bound-l}相加我们能够得到最外层的循环边界:
\textcolor{dkgreen}{$$2 \le m < L + D -1$$}

从\eqref{eq:bound-l}我们能得到:$m - L < p \le m -1$,即:$m -L+1 \le p < m$;同时,$p$必须满足\eqref{eq:bound-d}的约束。这两个不等式求交集就是$p$的上下界:
\textcolor{dkgreen}{$$\text{max}(1, m -L+1) \le p < \text{min}(m, D)$$}

\textcolor{dkgreen}{绿色}标注的三个不等式就是变换后循环的上下界,我们会得到下面这样一个变化后的程序:
\begin{lstlisting}[language=cplus]
for (:$m$:) in range (:$[2, L + D - 1)$:)  // 串行
  {  // 把所有可并行的数据batch成一个更大的数据,插入对gather_nd的调用
    for (:$n$:) in range (:$[0, N)$:)  // 并行
      for (:$p$:) in range (:[$\max(1, m - L + 1),\ \ \min(m, D))$:) // 并行
          batched_data = gather(可并行的数据点)
  }
  UDF(batched_data)
\end{lstlisting}
