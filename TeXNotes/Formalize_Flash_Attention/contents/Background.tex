A \textit{list} is a linearly ordered collection of values. All the elements of a given list must have the same type.
If the elements of a list all have the type $\alpha$, then the list itself has the type of $[\alpha]$.
% A list is either empty, $[\ ]$, or can be thought of as having a first element, or \textbf{head}, $a$ and a remainder, or \textbf{x}, $x$. The list with head $a$ and tail $x$ is written $(a:x)$.
% "$:$" is read as "cons" which appends an element at the start of another list.

\subsection{The \textit{reduce} operator}

The \textbf{\textit{reduce}} operator is a high-order list processing function where a list is \textcolor{red}{aggregated and reduced down to a single value}.
It is defined as follows:

\begin{align*}
&\textit{reduce} :: (\alpha \rightarrow \beta \rightarrow \beta) \rightarrow \beta \rightarrow ([\alpha] \rightarrow \beta) \\
&\textit{reduce} \ \oplus\ v \ [] = v \\
&\textit{reduce}\ \oplus\ v\ (x:xs) = \oplus \ x\ (\textit{reduce}\ \oplus\ v\ xs)
\end{align*}

The reduce operator takes a binary operator $\oplus$ of type $\alpha \rightarrow \beta \rightarrow \beta$, an initial value $v$ of type $\beta$, and a list of values $xs$ of type $[\alpha]$ as input,
and returns a single value of $\beta$ obtained by applying $\oplus$ to the initial value and the first element of the list,
then to the result of that operation and the second element of the list, and so on, until all elements have been processed.

\subsection{The \textit{map} operator}
\begin{align*}
\textit{map} &:: (\alpha \rightarrow \beta) \rightarrow ([\alpha] \rightarrow [\beta]) \\
\textit{map}\ f \ xs &= \left[ f \ x_1 \ldots f\ x_n\right]
\end{align*}

The \textit{map} operation applies a unary function $f$ of type $\alpha \rightarrow \beta$ to each element of a list.
Because there is no way to communicate among each evaluation of $f$, the underlying implementation of the \textit{map} operation can execute the evaluations of $f$ over list elements in any order it chooses.