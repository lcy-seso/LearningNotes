

\section{the $2\mathbf{d}+1$ representation}

\subsection {Classical loop transformations}

\begin{enumerate}
  \item only modif the iteration domain but do not affact the order in which
  statement instances are executed or the way arrays are accessed;
  \begin{enumerate}
    \item loop unrolling
    \item strip-mining
  \end{enumerate}
  \item modifies both iteration domain and a schedule transformation;
  \begin{enumerate}
    \item tiling: a combination of strip-mining and loop interchange;
  \end{enumerate}
  \item modifies schedule;
  \begin{enumerate}
    \item shifting/pipelining
  \end{enumerate}
  \item modifies array subscripts
  \begin{enumerate}
    \item privatization
  \end{enumerate}
  \item only modifies the array declarations (data layout)
  \begin{enumerate}
    \item padding
  \end{enumerate}
\end{enumerate}

\subsection{Polyhedral model}
Refer to paper \cite{girbal2006semi}\cite{vasilache2007scalable} for details.

The polyhedral representation is a semantics-based representation instead of
syntax-based representation. It clearly separates the four different types of
actions performed by program transformations:

\begin{enumerate}
  \item modification of the iteration domain (loop bounds and strides);
  \item modification of the schedule of each individual statement;
  \item modification of access functions (array subscripts)
  \item modification of the data layout (array declarations)
\end{enumerate}

Loop transformations are expressed as a "syntax-free" function compositions.

Aribitrarily complex compositions of classical transformations can be captured
in one single transformation step of the polyhedral model.
