\tolerance=1
\emergencystretch=\maxdimen
\hyphenpenalty=10000
\hbadness=10000

%\usepackage[document]{ragged2e}  % uncomment to enable left-alignment.
\usepackage{amsmath, amsfonts, stmaryrd, amssymb} % Math packages
\usepackage{iitem}
\usepackage{url}
\usepackage{cite}
\usepackage{bm}
\usepackage{booktabs}

\usepackage{soul}
% \setulcolor{red}

% \makeatletter
% \def\@cite#1#2 {\textsuperscript {[{#1\if@tempswa, #2\fi}]}}
% \makeatother
\usepackage{amsthm}

\usepackage{fourier}  % Use the Fourier fonts
%\usepackage{XCharter} % Use the XCharter fonts
%\usepackage{verbatim}
%\usepackage{mathpazo}

\usepackage{setspace}

\usepackage[rgb, dvipsnames]{xcolor}
\definecolor{dkgreen}{RGB}{0, 139, 139}
\definecolor{springgreen}{RGB}{60, 179, 113}
\definecolor{pg}{RGB}{84, 139, 84}
\definecolor{og}{RGB}{85,107,47}
\definecolor{vr}{RGB}{208, 32, 144}
\definecolor{dsgray}{RGB}{47, 79, 79}
\definecolor{slateblue}{RGB}{106, 90, 205}
\definecolor{skyblue}{RGB}{135, 206, 235}

\usepackage[framemethod=tikz]{mdframed}

\usepackage[
  colorlinks=true,
  linkcolor=blue,
  filecolor=blue,
  urlcolor=dkgreen,
  anchorcolor=blue,
  citecolor=blue]{hyperref}

\usepackage{enumerate} % Custom item numbers for enumerations
\usepackage{enumitem}
\setenumerate[1]{itemsep=0pt,partopsep=0pt,parsep=\parskip,topsep=4pt}
\setitemize[1]{itemsep=0pt,partopsep=0pt,parsep=\parskip,topsep=4pt}
\setdescription{itemsep=0pt,partopsep=0pt,parsep=\parskip,topsep=4pt}
\setdescription{itemsep=0pt,partopsep=0pt,parsep=\parskip,topsep=4pt}

\usepackage[ruled]{algorithm2e} % Algorithms
\usepackage[framemethod = tikz]{mdframed} % Allows defining custom boxed/framed environments
\usepackage{listings} % File listings, with syntax highlighting

\lstset{
language=Python,  % the language of the code
    basicstyle=\footnotesize,  % the size of the fonts that are used for the code
    numbers=left,  % where to put the line - numbers
    numberstyle=\tiny\color {gray},  % the style that is used for the line - numbers
    stepnumber=1,  % the step between two line - numbers. If it's 1, each line will be numbered
    numbersep=5pt,                  % how far the line-numbers are from the code
    backgroundcolor=\color{white},      % choose the background color. You must add \usepackage{color}
    showspaces=false,               % show spaces adding particular underscores
    showstringspaces=false,         % underline spaces within strings
    showtabs=false,                 % show tabs within strings adding particular underscores
    frame=single,                   % adds a frame around the code
    rulecolor=\color{black},        % if not set, the frame-color may be changed on
                                    % line-breaks within not-black text (e.g. commens (green here))
    tabsize=2,                      % sets default tabsize to 2 spaces
    captionpos=b,                   % sets the caption-position to bottom
    breaklines=true,                % sets automatic line breaking
    breakatwhitespace=false,        % sets if automatic breaks should only happen at whitespace
    title=\lstname,                 % show the filename of files included with \lstinputlisting;
                                    % also try caption instead of title
    keywordstyle=\color{blue},          % keyword style
    commentstyle=\color{dkgreen},       % comment style
    stringstyle=\color{gray},         % string literal style
}

%-------------------------------------------------------------------------------
%  DOCUMENT MARGINS
%-------------------------------------------------------------------------------

\usepackage{indentfirst}
\selectfont
\setlength{\topskip}{0ex}
\setlength{\lineskip}{1em}
\linespread{1.1}
\setlength{\parskip}{0.2em}
\setlength\parindent{0pt} % Removes all indentation from paragraphs

\usepackage{geometry} % Required for adjusting page dimensions and margins

\geometry{
  paper=a4paper, % Paper size, change to letterpaper for US letter size
  top=2.5cm, % Top margin
  bottom=3cm, % Bottom margin
  left=2.5cm, % Left margin
  right=2.5cm, % Right margin
  headheight=14pt, % Header height
  footskip=1.5cm, % Space from the bottom margin to the baseline of the footer
  headsep=1.2cm, % Space from the top margin to the baseline of the header
  % showframe, % Uncomment to show how the type block is set on the page
}

%----------------------------------------------------------------------------------------
%  FONTS
%----------------------------------------------------------------------------------------
\usepackage[utf8]{inputenc} % Required for inputting international characters
\usepackage[T1]{fontenc} % Output font encoding for international characters

%----------------------------------------------------------------------------------------
%  COMMAND LINE ENVIRONMENT
%----------------------------------------------------------------------------------------

% Usage:
% \begin{commandline}
%  \begin{verbatim}
%    $ ls
%
%    Applications   Desktop ...
%  \end{verbatim}
% \end{commandline}

\mdfdefinestyle{commandline}{
  leftmargin=10pt,
  rightmargin=10pt,
  innerleftmargin=15pt,
  middlelinecolor=black!50!white,
  middlelinewidth=2pt,
  frametitlerule=false,
  backgroundcolor=black!5!white,
  frametitle={Command Line},
  frametitlefont={\normalfont\sffamily\color{white}\hspace{-1em}},
  frametitlebackgroundcolor=black!50!white,
  nobreak,
}

% Define a custom environment for command-line snapshots
\newenvironment{commandline}{
  \medskip
  \begin{mdframed}[style=commandline]
}{
  \end{mdframed}
  \medskip
}

%----------------------------------------------------------------------------------------
%  FILE CONTENTS ENVIRONMENT
%----------------------------------------------------------------------------------------

% Usage:
% \begin{file}[optional filename, defaults to "File"]
%  File contents, for example, with a listings environment
% \end{file}

\mdfdefinestyle{file}{
  innertopmargin=1.6\baselineskip,
  innerbottommargin=0.8\baselineskip,
  topline=false, bottomline=false,
  leftline=false, rightline=false,
  leftmargin=2cm,
  rightmargin=2cm,
  singleextra={%
    \draw[fill=black!10!white](P)++(0,-1.2em)rectangle(P-|O);
    \node[anchor=north west]
    at(P-|O){\ttfamily\mdfilename};
    %
    \def\l{3em}
    \draw(O-|P)++(-\l,0)--++(\l,\l)--(P)--(P-|O)--(O)--cycle;
    \draw(O-|P)++(-\l,0)--++(0,\l)--++(\l,0);
  },
  nobreak,
}

% Define a custom environment for file contents
\newenvironment{file}[1][File]{ % Set the default filename to "File"
  \medskip
  \newcommand{\mdfilename}{#1}
  \begin{mdframed}[style=file]
}{
  \end{mdframed}
  \medskip
}

%----------------------------------------------------------------------------------------
%  NUMBERED QUESTIONS ENVIRONMENT
%----------------------------------------------------------------------------------------

% Usage:
% \begin{question}[optional title]
%  Question contents
% \end{question}

\mdfdefinestyle{question}{
  innertopmargin=1.2\baselineskip,
  innerbottommargin=0.8\baselineskip,
  roundcorner=5pt,
  nobreak,
  singleextra={%
    \draw(P-|O)node[xshift=1em,anchor=west,fill=white,draw,rounded corners=5pt]{%
    Question \theQuestion\questionTitle};
  },
}

\newcounter{Question} % Stores the current question number that gets iterated with each new question

% Define a custom environment for numbered questions
\newenvironment{question}[1][\unskip]{
  \bigskip
  \stepcounter{Question}
  \newcommand{\questionTitle}{~#1}
  \begin{mdframed}[style=question]
}{
  \end{mdframed}
  \medskip
}

%----------------------------------------------------------------------------------------
%  WARNING TEXT ENVIRONMENT
%----------------------------------------------------------------------------------------

% Usage:
% \begin{warn}[optional title, defaults to "Warning:"]
%  Contents
% \end{warn}

\mdfdefinestyle{warning}{
  topline=false, bottomline=false,
  leftline=false, rightline=false,
  nobreak,
  singleextra={%
    \draw(P-|O)++(-0.5em,0)node(tmp1){};
    \draw(P-|O)++(0.5em,0)node(tmp2){};
    \fill[black,rotate around={45:(P-|O)}](tmp1)rectangle(tmp2);
    \node at(P-|O){\color{white}\scriptsize\bf !};
    \draw[very thick](P-|O)++(0,-1em)--(O);%--(O-|P);
  }
}

% Define a custom environment for warning text
\newenvironment{warn}[1][Warning:]{ % Set the default warning to "Warning:"
  \medskip
  \begin{mdframed}[style=warning]
    \noindent{\textbf{#1}}
}{
  \end{mdframed}
}

%----------------------------------------------------------------------------------------
%  INFORMATION ENVIRONMENT
%----------------------------------------------------------------------------------------

% Usage:
% \begin{info}[optional title, defaults to "Info:"]
%   contents
%   \end{info}

\mdfdefinestyle{info}{%
  topline=false, bottomline=false,
  leftline=false, rightline=false,
  nobreak,
  singleextra={%
    \fill[black](P-|O)circle[radius=0.4em];
    \node at(P-|O){\color{white}\scriptsize\bf i};
    \draw[very thick](P-|O)++(0,-0.8em)--(O);%--(O-|P);
  }
}

% Define a custom environment for information
\newenvironment{info}[1][Info:]{ % Set the default title to "Info:"
  \medskip
  \begin{mdframed}[style=info]
    \noindent{\textbf{#1}}
}{
  \end{mdframed}
}
