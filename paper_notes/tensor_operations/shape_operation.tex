
\section{TensorShape operations}

\subsection{Basic query}

\subsubsection{\textbf{\textit{size}}}

\textbf{\textit{size}}: returns a tuple of tensor $A$'s size, or the size of a given dimension.
\begin{lstlisting}[language=Python]
size(A:Tensor[T]) -> Tuple[int]
size(A:Tensor[T], dim:int) -> int
\end{lstlisting}

\subsubsection{\textbf{\textit{numel}}}

\textbf{\textit{numel}}: returns the cardinality of tensor $A$. It is equivalent to prod(size($A$)).
\begin{lstlisting}[language=Python]
numel(tup:Tuple[int]) -> int
\end{lstlisting}

\subsubsection{\textbf{\textit{ndims}}}

\textbf{\textit{ndims}}: returns number of dimensions of $A$.
\begin{lstlisting}[language=Python]
ndims(tup:Tuple[int]) -> int
\end{lstlisting}

\subsection{Element updating}

\begin{itemize}
  \item Tuples are immutable so its value cannot be updated or changed.
  \item A new tuple canbe created from a tuple.
\end{itemize}

\subsubsection{\textbf{\textit{add}}}
  \begin{lstlisting}[language=Python]
  add(tup1:Tuple[int], tup2:Tuple[int]) ->Tuple[int]
  \end{lstlisting}
  Examples:
  \begin{lstlisting}[language=Python]
  tup1 = (1, 2, 3)
  tup2 = (4, 5)
  tup1 + tup2 = (1, 2, 3, 4, 5)
  \end{lstlisting}

\subsubsection{\textbf{\textit{insert}}}

insert an element into a tuple at a given position

\begin{lstlisting}[language=Python]
insert(tup:Tuple[int], pos:int, value:int) -> Tuple[int]
\end{lstlisting}

Example:
\begin{lstlisting}[language=Python]
tup = (1, 2, 3)
insert(tup, 1, 5)
ans = (1, 5, 2, 3)
\end{lstlisting}

\subsubsection{\textbf{\textit{del}}}

remove tuple elements by index
\begin{lstlisting}[language=Python]
del(tup:Tuple[int], pos:int) -> Tuple[int]
\end{lstlisting}
Example:
\begin{lstlisting}[language=Python]
tup = (2, 3, 4)
del(tup, 1)
ans = (2, 4)
\end{lstlisting}

\subsubsection{\textbf{\textit{replace}}}

replace the $i$-th tuple element with a given value. \textit{\textbf{replace}} is a combination of \textit{\textbf{del}} and \textit{\textbf{insert}}.

\begin{lstlisting}[language=Python]
replace(tup:Tuple[int], pos:int, value:int) -> Tuple[int]
\end{lstlisting}

Example:
\begin{lstlisting}[language=Python]
tup = (2, 3, 4)
replace(tup, 1, 5)
ans = (2, 5, 4)
\end{lstlisting}

\subsection{Index generation}

\subsubsection{\textbf{\textit{cartesian\_product}}}

\begin{lstlisting}[language=Python]
cartesian_product(*tup:Tuple[int]) -> Tuple[Tuple[int]]
\end{lstlisting}

Examples:
\begin{lstlisting}[language=Python]
tup1 = (2, 3, 4)
tup2 = (0, 1)
cartesian_product(tup1, tup2)
ans = ((2, 0), (3, 0), (4, 0), (2, 1), (3, 1), (4, 1))
\end{lstlisting}

\subsubsection{\textbf{\textit{meshgrid}}}

\subsubsection{\textbf{\textit{arange}}}


\subsection{Shape function}

Shape function recieves shapes of input operands and optional arguments, such as a specified dimension, stride, etc. and returns the shape of the output.

$$S(Y) = \Gamma(S(X_i), \text{**kwargs})$$ where $i = 0,...,N$ and $N$ is the number of operands.
